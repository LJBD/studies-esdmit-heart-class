\section{Podsumowanie}

\qquad W ramach projektu zaimplementowano moduł HeartClass w trzech różnych językach programowania: Matlabie, Pythonie (3.4.x) oraz Julii (0.4.x). Moduł ten ma celu klasyfikowanie zespołów QRS sygnału EKG. W pracy udało się zaimplementować moduł odpowiedzialny za klasteryzację oraz klasyfikację. Z wykonanych testów wynika, że moduł najszybciej wykonuje się w Matlabie. Powodem tego jest głównie fakt, iż w tym module wykorzystano gotową bibliotekę, a nie, jak w przypadku pozostałych dwóch, gdzie zaimplementowano własną maszynę SVM. Dużym zaskoczeniem okazała się implementacja w Julii. Moduł średnio wykonuje się prawie 10 razy szybciej niż w Pythonie i niewiele wolniej niż w Matlabie.

Ciężko dokonać jakiegokolwiek sensownego porównania z bazową implementacją w języku C++. Jest to, oczywiście, spowodowane jej wybrakowaniem, szczególnie w kontekście implementacji algorytmu G-średnich. Co jednak trzeba jej oddać, algorytm SVM funkcjonuje tam zdecydowanie lepiej. Jest to niejakim zaskoczeniem, ponieważ wszystkie 3 implementacje korzystają z tej samej biblioteki i tego samego modelu.

