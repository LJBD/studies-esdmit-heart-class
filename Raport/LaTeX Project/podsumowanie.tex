\section{Podsumowanie}

W ramach projektu zaimplementowano moduł HeartClass w trzech różnych językach programowania: Matlabie, Pythonie (3.4.x) oraz Julii (0.4.x). Moduł ten ma celu klasyfikowanie zespołów QRS sygnału EKG. W pracy udało się zaimplementować moduł odpowiedzialny za klasteryzację oraz klasyfikację. Z wykonanych testów wynika, że moduł najszybciej wykonuje się w Matlabie. Powodem tego jest głównie fakt, iż w tym module wykorzystano gotową bibliotekę, a nie, jak w przypadku pozostałych dwóch, gdzie zaimplementowano własną maszynę SVM. Dużym zaskoczeniem okazała się implementacja w Julii. Moduł średnio wykonuje się prawie 10 razy szybciej niż w Pythonie i nie wiele wolniej niż w Matlabie.

